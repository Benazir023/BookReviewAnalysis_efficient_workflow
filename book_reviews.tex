% Options for packages loaded elsewhere
\PassOptionsToPackage{unicode}{hyperref}
\PassOptionsToPackage{hyphens}{url}
%
\documentclass[
]{article}
\usepackage{amsmath,amssymb}
\usepackage{lmodern}
\usepackage{iftex}
\ifPDFTeX
  \usepackage[T1]{fontenc}
  \usepackage[utf8]{inputenc}
  \usepackage{textcomp} % provide euro and other symbols
\else % if luatex or xetex
  \usepackage{unicode-math}
  \defaultfontfeatures{Scale=MatchLowercase}
  \defaultfontfeatures[\rmfamily]{Ligatures=TeX,Scale=1}
\fi
% Use upquote if available, for straight quotes in verbatim environments
\IfFileExists{upquote.sty}{\usepackage{upquote}}{}
\IfFileExists{microtype.sty}{% use microtype if available
  \usepackage[]{microtype}
  \UseMicrotypeSet[protrusion]{basicmath} % disable protrusion for tt fonts
}{}
\makeatletter
\@ifundefined{KOMAClassName}{% if non-KOMA class
  \IfFileExists{parskip.sty}{%
    \usepackage{parskip}
  }{% else
    \setlength{\parindent}{0pt}
    \setlength{\parskip}{6pt plus 2pt minus 1pt}}
}{% if KOMA class
  \KOMAoptions{parskip=half}}
\makeatother
\usepackage{xcolor}
\usepackage[margin=1in]{geometry}
\usepackage{color}
\usepackage{fancyvrb}
\newcommand{\VerbBar}{|}
\newcommand{\VERB}{\Verb[commandchars=\\\{\}]}
\DefineVerbatimEnvironment{Highlighting}{Verbatim}{commandchars=\\\{\}}
% Add ',fontsize=\small' for more characters per line
\usepackage{framed}
\definecolor{shadecolor}{RGB}{248,248,248}
\newenvironment{Shaded}{\begin{snugshade}}{\end{snugshade}}
\newcommand{\AlertTok}[1]{\textcolor[rgb]{0.94,0.16,0.16}{#1}}
\newcommand{\AnnotationTok}[1]{\textcolor[rgb]{0.56,0.35,0.01}{\textbf{\textit{#1}}}}
\newcommand{\AttributeTok}[1]{\textcolor[rgb]{0.77,0.63,0.00}{#1}}
\newcommand{\BaseNTok}[1]{\textcolor[rgb]{0.00,0.00,0.81}{#1}}
\newcommand{\BuiltInTok}[1]{#1}
\newcommand{\CharTok}[1]{\textcolor[rgb]{0.31,0.60,0.02}{#1}}
\newcommand{\CommentTok}[1]{\textcolor[rgb]{0.56,0.35,0.01}{\textit{#1}}}
\newcommand{\CommentVarTok}[1]{\textcolor[rgb]{0.56,0.35,0.01}{\textbf{\textit{#1}}}}
\newcommand{\ConstantTok}[1]{\textcolor[rgb]{0.00,0.00,0.00}{#1}}
\newcommand{\ControlFlowTok}[1]{\textcolor[rgb]{0.13,0.29,0.53}{\textbf{#1}}}
\newcommand{\DataTypeTok}[1]{\textcolor[rgb]{0.13,0.29,0.53}{#1}}
\newcommand{\DecValTok}[1]{\textcolor[rgb]{0.00,0.00,0.81}{#1}}
\newcommand{\DocumentationTok}[1]{\textcolor[rgb]{0.56,0.35,0.01}{\textbf{\textit{#1}}}}
\newcommand{\ErrorTok}[1]{\textcolor[rgb]{0.64,0.00,0.00}{\textbf{#1}}}
\newcommand{\ExtensionTok}[1]{#1}
\newcommand{\FloatTok}[1]{\textcolor[rgb]{0.00,0.00,0.81}{#1}}
\newcommand{\FunctionTok}[1]{\textcolor[rgb]{0.00,0.00,0.00}{#1}}
\newcommand{\ImportTok}[1]{#1}
\newcommand{\InformationTok}[1]{\textcolor[rgb]{0.56,0.35,0.01}{\textbf{\textit{#1}}}}
\newcommand{\KeywordTok}[1]{\textcolor[rgb]{0.13,0.29,0.53}{\textbf{#1}}}
\newcommand{\NormalTok}[1]{#1}
\newcommand{\OperatorTok}[1]{\textcolor[rgb]{0.81,0.36,0.00}{\textbf{#1}}}
\newcommand{\OtherTok}[1]{\textcolor[rgb]{0.56,0.35,0.01}{#1}}
\newcommand{\PreprocessorTok}[1]{\textcolor[rgb]{0.56,0.35,0.01}{\textit{#1}}}
\newcommand{\RegionMarkerTok}[1]{#1}
\newcommand{\SpecialCharTok}[1]{\textcolor[rgb]{0.00,0.00,0.00}{#1}}
\newcommand{\SpecialStringTok}[1]{\textcolor[rgb]{0.31,0.60,0.02}{#1}}
\newcommand{\StringTok}[1]{\textcolor[rgb]{0.31,0.60,0.02}{#1}}
\newcommand{\VariableTok}[1]{\textcolor[rgb]{0.00,0.00,0.00}{#1}}
\newcommand{\VerbatimStringTok}[1]{\textcolor[rgb]{0.31,0.60,0.02}{#1}}
\newcommand{\WarningTok}[1]{\textcolor[rgb]{0.56,0.35,0.01}{\textbf{\textit{#1}}}}
\usepackage{graphicx}
\makeatletter
\def\maxwidth{\ifdim\Gin@nat@width>\linewidth\linewidth\else\Gin@nat@width\fi}
\def\maxheight{\ifdim\Gin@nat@height>\textheight\textheight\else\Gin@nat@height\fi}
\makeatother
% Scale images if necessary, so that they will not overflow the page
% margins by default, and it is still possible to overwrite the defaults
% using explicit options in \includegraphics[width, height, ...]{}
\setkeys{Gin}{width=\maxwidth,height=\maxheight,keepaspectratio}
% Set default figure placement to htbp
\makeatletter
\def\fps@figure{htbp}
\makeatother
\setlength{\emergencystretch}{3em} % prevent overfull lines
\providecommand{\tightlist}{%
  \setlength{\itemsep}{0pt}\setlength{\parskip}{0pt}}
\setcounter{secnumdepth}{-\maxdimen} % remove section numbering
\ifLuaTeX
  \usepackage{selnolig}  % disable illegal ligatures
\fi
\IfFileExists{bookmark.sty}{\usepackage{bookmark}}{\usepackage{hyperref}}
\IfFileExists{xurl.sty}{\usepackage{xurl}}{} % add URL line breaks if available
\urlstyle{same} % disable monospaced font for URLs
\hypersetup{
  pdftitle={Book Reviews},
  pdfauthor={Bena},
  hidelinks,
  pdfcreator={LaTeX via pandoc}}

\title{Book Reviews}
\author{Bena}
\date{2023-04-03}

\begin{document}
\maketitle

As a data analyst for a company that sells books for learning
programming, your company has produced multiple books, and each has
received many reviews. The company wants us to check out the sales data
and see if we can extract any useful information from it.

\begin{Shaded}
\begin{Highlighting}[]
\FunctionTok{library}\NormalTok{(tidyverse)}
\end{Highlighting}
\end{Shaded}

\begin{verbatim}
## -- Attaching packages --------------------------------------- tidyverse 1.3.2 --
## v ggplot2 3.4.0     v purrr   1.0.1
## v tibble  3.1.8     v dplyr   1.1.0
## v tidyr   1.3.0     v stringr 1.5.0
## v readr   2.1.3     v forcats 1.0.0
## -- Conflicts ------------------------------------------ tidyverse_conflicts() --
## x dplyr::filter() masks stats::filter()
## x dplyr::lag()    masks stats::lag()
\end{verbatim}

\hypertarget{import-data}{%
\subsection{Import data}\label{import-data}}

\begin{Shaded}
\begin{Highlighting}[]
\NormalTok{book\_reviews }\OtherTok{\textless{}{-}} \FunctionTok{read\_csv}\NormalTok{(}\StringTok{"D:/BENA/Data Analytics/Dataquest/Project2\_DataCleaning/book\_reviews.csv"}\NormalTok{)}
\end{Highlighting}
\end{Shaded}

\begin{verbatim}
## Rows: 2000 Columns: 4
## -- Column specification --------------------------------------------------------
## Delimiter: ","
## chr (3): book, review, state
## dbl (1): price
## 
## i Use `spec()` to retrieve the full column specification for this data.
## i Specify the column types or set `show_col_types = FALSE` to quiet this message.
\end{verbatim}

\hypertarget{about-data}{%
\subsection{About data}\label{about-data}}

\begin{Shaded}
\begin{Highlighting}[]
\FunctionTok{dim}\NormalTok{(book\_reviews)}
\end{Highlighting}
\end{Shaded}

\begin{verbatim}
## [1] 2000    4
\end{verbatim}

\begin{Shaded}
\begin{Highlighting}[]
\FunctionTok{glimpse}\NormalTok{(book\_reviews)}
\end{Highlighting}
\end{Shaded}

\begin{verbatim}
## Rows: 2,000
## Columns: 4
## $ book   <chr> "R Made Easy", "R For Dummies", "R Made Easy", "R Made Easy", "~
## $ review <chr> "Excellent", "Fair", "Excellent", "Poor", "Great", NA, "Great",~
## $ state  <chr> "TX", "NY", "NY", "FL", "Texas", "California", "Florida", "CA",~
## $ price  <dbl> 19.99, 15.99, 19.99, 19.99, 50.00, 19.99, 19.99, 19.99, 29.99, ~
\end{verbatim}

\begin{Shaded}
\begin{Highlighting}[]
\FunctionTok{colnames}\NormalTok{(book\_reviews)}
\end{Highlighting}
\end{Shaded}

\begin{verbatim}
## [1] "book"   "review" "state"  "price"
\end{verbatim}

\begin{Shaded}
\begin{Highlighting}[]
\FunctionTok{typeof}\NormalTok{(book\_reviews)}
\end{Highlighting}
\end{Shaded}

\begin{verbatim}
## [1] "list"
\end{verbatim}

Note that the \texttt{echo\ =\ FALSE} parameter was added to the code
chunk to prevent printing of the R code that generated the plot.

\hypertarget{data-type-for-each-column}{%
\subsection{Data type for each column}\label{data-type-for-each-column}}

\begin{Shaded}
\begin{Highlighting}[]
\ControlFlowTok{for}\NormalTok{ (col }\ControlFlowTok{in} \FunctionTok{colnames}\NormalTok{(book\_reviews)) \{}
 \FunctionTok{print}\NormalTok{(}\FunctionTok{typeof}\NormalTok{(book\_reviews[[col]]))}
\NormalTok{\}}
\end{Highlighting}
\end{Shaded}

\begin{verbatim}
## [1] "character"
## [1] "character"
## [1] "character"
## [1] "double"
\end{verbatim}

\hypertarget{unique-values-in-each-column}{%
\subsection{Unique values in each
column}\label{unique-values-in-each-column}}

\begin{Shaded}
\begin{Highlighting}[]
\ControlFlowTok{for}\NormalTok{ (val }\ControlFlowTok{in} \FunctionTok{colnames}\NormalTok{(book\_reviews)) \{}
  \FunctionTok{print}\NormalTok{(}\StringTok{"unique values"}\NormalTok{)}
  \FunctionTok{print}\NormalTok{(val)}
  \FunctionTok{print}\NormalTok{(}\FunctionTok{unique}\NormalTok{(book\_reviews[[val]]))}
\NormalTok{\}}
\end{Highlighting}
\end{Shaded}

\begin{verbatim}
## [1] "unique values"
## [1] "book"
## [1] "R Made Easy"                        "R For Dummies"                     
## [3] "Secrets Of R For Advanced Students" "Top 10 Mistakes R Beginners Make"  
## [5] "Fundamentals of R For Beginners"   
## [1] "unique values"
## [1] "review"
## [1] "Excellent" "Fair"      "Poor"      "Great"     NA          "Good"     
## [1] "unique values"
## [1] "state"
## [1] "TX"         "NY"         "FL"         "Texas"      "California"
## [6] "Florida"    "CA"         "New York"  
## [1] "unique values"
## [1] "price"
## [1] 19.99 15.99 50.00 29.99 39.99
\end{verbatim}

\hypertarget{handling-missing-data}{%
\subsection{Handling missing data}\label{handling-missing-data}}

\begin{Shaded}
\begin{Highlighting}[]
\NormalTok{book\_reviews\_nonulls }\OtherTok{\textless{}{-}}\NormalTok{ book\_reviews }\SpecialCharTok{\%\textgreater{}\%}
  \FunctionTok{filter}\NormalTok{(}\SpecialCharTok{!}\FunctionTok{is.na}\NormalTok{(review))}

\FunctionTok{dim}\NormalTok{(book\_reviews\_nonulls)}
\end{Highlighting}
\end{Shaded}

\begin{verbatim}
## [1] 1794    4
\end{verbatim}

\begin{Shaded}
\begin{Highlighting}[]
\FunctionTok{View}\NormalTok{(book\_reviews\_nonulls)}
\end{Highlighting}
\end{Shaded}

\hypertarget{investigate-missing-data}{%
\subsection{Investigate missing data}\label{investigate-missing-data}}

\begin{Shaded}
\begin{Highlighting}[]
\NormalTok{book\_reviews\_nulls }\OtherTok{\textless{}{-}}\NormalTok{ book\_reviews }\SpecialCharTok{\%\textgreater{}\%}
  \FunctionTok{filter}\NormalTok{(}\FunctionTok{is.na}\NormalTok{(review))}

\FunctionTok{View}\NormalTok{(book\_reviews\_nulls)}

\NormalTok{nulls\_per\_book }\OtherTok{\textless{}{-}}\NormalTok{ book\_reviews\_nulls }\SpecialCharTok{\%\textgreater{}\%}
  \FunctionTok{group\_by}\NormalTok{(book) }\SpecialCharTok{\%\textgreater{}\%}
  \FunctionTok{summarise}\NormalTok{(}
    \AttributeTok{nulls =} \FunctionTok{n}\NormalTok{()}
\NormalTok{  ) }\SpecialCharTok{\%\textgreater{}\%}
  \FunctionTok{arrange}\NormalTok{(}\SpecialCharTok{{-}}\NormalTok{nulls)}

\NormalTok{nulls\_per\_state }\OtherTok{\textless{}{-}}\NormalTok{ book\_reviews\_nulls }\SpecialCharTok{\%\textgreater{}\%}
  \FunctionTok{group\_by}\NormalTok{(state)}\SpecialCharTok{\%\textgreater{}\%}
  \FunctionTok{summarise}\NormalTok{(}
    \AttributeTok{nulls =} \FunctionTok{n}\NormalTok{()}
\NormalTok{  ) }\SpecialCharTok{\%\textgreater{}\%}
  \FunctionTok{arrange}\NormalTok{(}\SpecialCharTok{{-}}\NormalTok{nulls)}
\end{Highlighting}
\end{Shaded}

Some 206 rows with nulls in the reviews column were deleted. We still
have 89.7\% of our data available.

Nulls per book are: 1 Fundamentals of R For Beginners 44 2 R For Dummies
49 3 R Made Easy 37 4 Secrets Of R For Advanced Students 46 5 Top 10
Mistakes R Beginners Make 30

Nulls per state are: 1 TX 62 2 CA 54 3 NY 47 4 FL 43

\hypertarget{standardize-values-in-state-column}{%
\subsection{Standardize values in state
column}\label{standardize-values-in-state-column}}

\begin{Shaded}
\begin{Highlighting}[]
\NormalTok{book\_reviews\_nonulls }\OtherTok{\textless{}{-}}\NormalTok{ book\_reviews\_nonulls }\SpecialCharTok{\%\textgreater{}\%}
  \FunctionTok{mutate}\NormalTok{(}
    \AttributeTok{state =} \FunctionTok{case\_when}\NormalTok{(}
\NormalTok{      state }\SpecialCharTok{==} \StringTok{"Texas"} \SpecialCharTok{\textasciitilde{}} \StringTok{"TX"}\NormalTok{,}
\NormalTok{      state }\SpecialCharTok{==} \StringTok{"California"} \SpecialCharTok{\textasciitilde{}} \StringTok{"CA"}\NormalTok{,}
\NormalTok{      state }\SpecialCharTok{==} \StringTok{"Florida"} \SpecialCharTok{\textasciitilde{}} \StringTok{"FL"}\NormalTok{,}
\NormalTok{      state }\SpecialCharTok{==} \StringTok{"New York"} \SpecialCharTok{\textasciitilde{}} \StringTok{"NY"}\NormalTok{,}
      \ConstantTok{TRUE} \SpecialCharTok{\textasciitilde{}}\NormalTok{ state}
\NormalTok{    )}
\NormalTok{  )}

\NormalTok{book\_reviews\_nulls }\OtherTok{\textless{}{-}}\NormalTok{ book\_reviews\_nulls }\SpecialCharTok{\%\textgreater{}\%}
  \FunctionTok{mutate}\NormalTok{(}
    \AttributeTok{state =} \FunctionTok{case\_when}\NormalTok{(}
\NormalTok{      state }\SpecialCharTok{==} \StringTok{"Texas"} \SpecialCharTok{\textasciitilde{}} \StringTok{"TX"}\NormalTok{,}
\NormalTok{      state }\SpecialCharTok{==} \StringTok{"California"} \SpecialCharTok{\textasciitilde{}} \StringTok{"CA"}\NormalTok{,}
\NormalTok{      state }\SpecialCharTok{==} \StringTok{"Florida"} \SpecialCharTok{\textasciitilde{}} \StringTok{"FL"}\NormalTok{,}
\NormalTok{      state }\SpecialCharTok{==} \StringTok{"New York"} \SpecialCharTok{\textasciitilde{}} \StringTok{"NY"}\NormalTok{,}
      \ConstantTok{TRUE} \SpecialCharTok{\textasciitilde{}}\NormalTok{ state}
\NormalTok{    )}
\NormalTok{  )}
\end{Highlighting}
\end{Shaded}

Have a look at the results

\begin{Shaded}
\begin{Highlighting}[]
\FunctionTok{View}\NormalTok{(book\_reviews\_nonulls)}
\end{Highlighting}
\end{Shaded}

\hypertarget{transform-text-data-to-number-type-data}{%
\subsection{Transform text data to number type
data}\label{transform-text-data-to-number-type-data}}

\begin{Shaded}
\begin{Highlighting}[]
\NormalTok{book\_reviews\_nonulls }\OtherTok{\textless{}{-}}\NormalTok{ book\_reviews\_nonulls }\SpecialCharTok{\%\textgreater{}\%}
  \FunctionTok{mutate}\NormalTok{(}
    \AttributeTok{review\_num =} \FunctionTok{case\_when}\NormalTok{(}
\NormalTok{      review }\SpecialCharTok{==} \StringTok{"Poor"} \SpecialCharTok{\textasciitilde{}} \DecValTok{1}\NormalTok{,}
\NormalTok{      review }\SpecialCharTok{==} \StringTok{"Fair"} \SpecialCharTok{\textasciitilde{}} \DecValTok{2}\NormalTok{,}
\NormalTok{      review }\SpecialCharTok{==} \StringTok{"Good"} \SpecialCharTok{\textasciitilde{}} \DecValTok{3}\NormalTok{,}
\NormalTok{      review }\SpecialCharTok{==} \StringTok{"Great"} \SpecialCharTok{\textasciitilde{}} \DecValTok{4}\NormalTok{,}
\NormalTok{      review }\SpecialCharTok{==} \StringTok{"Excellent"} \SpecialCharTok{\textasciitilde{}} \DecValTok{5}
\NormalTok{    )}
\NormalTok{  )}

\NormalTok{book\_reviews\_nonulls }\OtherTok{\textless{}{-}}\NormalTok{ book\_reviews\_nonulls }\SpecialCharTok{\%\textgreater{}\%}
  \FunctionTok{mutate}\NormalTok{(}
    \AttributeTok{is\_high\_review =} \FunctionTok{if\_else}\NormalTok{(review\_num }\SpecialCharTok{\textgreater{}=} \DecValTok{4}\NormalTok{,}\ConstantTok{TRUE}\NormalTok{,}\ConstantTok{FALSE}\NormalTok{) }
\NormalTok{  )}
\end{Highlighting}
\end{Shaded}

\hypertarget{finding-the-most-profitable-book}{%
\subsection{Finding the most profitable
book}\label{finding-the-most-profitable-book}}

\begin{Shaded}
\begin{Highlighting}[]
\NormalTok{profitable\_book }\OtherTok{\textless{}{-}}\NormalTok{ book\_reviews\_nonulls }\SpecialCharTok{\%\textgreater{}\%}
  \FunctionTok{group\_by}\NormalTok{(book) }\SpecialCharTok{\%\textgreater{}\%}
  \FunctionTok{summarize}\NormalTok{(}
    \AttributeTok{total\_price =} \FunctionTok{sum}\NormalTok{(price)}
\NormalTok{  ) }\SpecialCharTok{\%\textgreater{}\%}
  \FunctionTok{arrange}\NormalTok{(}\SpecialCharTok{{-}}\NormalTok{total\_price)}
\end{Highlighting}
\end{Shaded}

Secrets of R For Advanced Students had the highest total price.
Alternatively, considering the number of books sold.

\begin{Shaded}
\begin{Highlighting}[]
\NormalTok{profitable\_book }\OtherTok{\textless{}{-}}\NormalTok{ book\_reviews\_nonulls }\SpecialCharTok{\%\textgreater{}\%}
  \FunctionTok{group\_by}\NormalTok{(book) }\SpecialCharTok{\%\textgreater{}\%}
  \FunctionTok{summarize}\NormalTok{(}
    \AttributeTok{no\_purchased =} \FunctionTok{n}\NormalTok{()}
\NormalTok{  ) }\SpecialCharTok{\%\textgreater{}\%}
  \FunctionTok{arrange}\NormalTok{(}\SpecialCharTok{{-}}\NormalTok{no\_purchased)}
\end{Highlighting}
\end{Shaded}

Fundamentals of R For Beginners had the highest number of purchases

\hypertarget{reporting-the-results}{%
\subsection{Reporting the results}\label{reporting-the-results}}

This analysis is motivated by the fact that the company wants us to
explore the book sales data and gain valuable insights from it. Therein,
we have quantitative data i.e.~price and qualitative data such as
reviews. More information is also provided i.e.~the book name and state
where the sale was made. The main question we're trying to answer is how
profitable are the book sales.

In the data preparation, some of the things that had to be done to make
it more usable include:  Converting data in the review column from
string to numeric i.e.~1 to 5, 5 being excellent  Filtering missing
data and reviewing it closely  Standardizing state names to respective
abbreviated code names eg from California to CA  Aggregating the data
using group by and count functions among others.  Sorting out the data
to find out which had the highest/lowest number of factor of interest eg
highest no. of books sold, ordering revenue earned from each book, book
with the most no. of favorable reviews.  Used control flow and logicals
to categorize the data.

In conclusion, the most profitable book in terms of revenue earned was
Secrets of R For Advanced Students \& the book with highest number sold
was Fundamentals of R For Beginners. More information could also be
provided that would aid in knowing what time of the year most sales are
made. It could also be helpful in determining whether or not to do a
sales \& marketing campaign and what time would be the best to do that.
These findings could be helpful to the stores \& procurement or
publishing section of the company. They're now more aware of which books
they should stock up more of.

\end{document}
